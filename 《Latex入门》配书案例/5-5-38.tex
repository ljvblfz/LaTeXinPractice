% \usepackage{pstricks-add}
% \usepackage{siunitx}
\begin{figure}
\centering
\newcommand\iangle{120}
\psset{unit=1.5cm,linewidth=0.4pt,algebraic=true}
\begin{pspicture}(-3.5,-1.5)(4.5,1.5)
\rput(-2,0){
  \psaxes[labels=none,ticks=none]{->}(0,0)(-1.2,-1.2)(1.2,1.2)
  \pscircle[linewidth=0.8pt](0,0){1}
  \pswedge[fillstyle=solid,fillcolor=gray,opacity=0.2]
    (0,0){1}{0}{\iangle}
  \pswedge[fillstyle=solid,fillcolor=gray,opacity=0.5]
    (0,0){0.3}{0}{\iangle}
  \uput[!\iangle\space 2 div]
    (0.3;!\iangle\space 2 div) {\ang{\iangle}}
  \pnode(1;\iangle){P}
  \pnode(P|0,0){P0}
  \ncline{-}{P}{P0}
  \uput[\iangle](P){$P$}
  \uput[d](P0){$P_0$}
}

\psaxes[labels=none,dx=1.57]
  {->}(0,0)(0,-1.2)(3.5,1.2)
\psplot[linewidth=0.8pt]{0}{3.5}{sin(x)}
\multido{\n=0+1.57,\i=0+90}{3}{
  \uput*[d](\n,0){\small\ang{\i}}
}
\uput[r](*{3.5} {sin(x)}){$\sin x$}
\pnode(!\iangle\space Pi mul 180 div \iangle\space sin){Q}
\pnode(Q|0,0){Q0}
\uput[u](Q){$Q$}
\uput[d](Q0){$Q_0$}
\ncline{-}{Q}{Q0}

\psline[linestyle=dashed](P)(Q)
\end{pspicture}
  \caption{正弦函数与单位圆(\textsf{PSTricks} 实现)}
  \label{fig:pstsine}
\end{figure}
